\section*{Acknowledgements}
	I would like to thank Dr. Miranda Mowbray for her guidance on both the computer science and law aspects of the project. I would also like to thank Dr. Megan Rae Blakely for sharing her knowledge of the law domain and for her feedback on the project.
	
	\newpage
\section*{Executive Summary}	
	This project is a contribution to Dr. Megan Rae Blakely's research into the growing relationship between intellectual property law and human rights law; in particular, the extent to which intellectual property laws involve human rights considerations and their balance between consideration of creators and users. Currently, the relationship between intellectual property and human rights is analysed on a case-by-case basis due to a lack of systematic means of analysis. The project shows that a systematic method in natural language processing can be appropriate in this domain and therefore allow a more comprehensive argument to be made about the relationship.
	
	The project shows the technology is appropriate for the domain through the use of three computer science skills: natural language processing, data visualisation and usability consideration. A support vector machine is used to analyse a set of journal articles to establish a model of what the language of human rights and intellectual property consists of. It is then used to test a further set of journal articles to see how the set compares to the model over time. A rules-based method is used to determine whether articles' tone suggest the legal context is benefiting the user or the creator. This data is then visualised in order to make the results of the natural language processing pallatable. This is done via an iterative process in order to find the most appropriate visualisation for the domain. A user interface is built in order for the intended user to be able to make the most of this functionality by adding new articles while not needing to know any of the intricacies of natural language processing.
	
	The natural language processing methods scored well in cross-validation using balanced accuracy scores and the final feedback from Dr. Blakely on the visualisation and user interface was positive. This was acknowledged at the British and Irish Law Education and Technology conference. 
	
	The tool will go on to be extended based on my recommended future work. The project will lead to further understanding of how past events have impacted the intellectual property field with respect to the human rights field and vice versa. Further, the project's success may encourage further use of technology in the domain.

\tableofcontents