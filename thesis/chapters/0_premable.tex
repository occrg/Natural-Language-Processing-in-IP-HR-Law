\section*{Acknowledgements}
	\newpage
	
% Background and motivation
% What is the problem? What do they do at the moment?
% Why is this a potential solution?
% What are the technical challenges from a Computer Science point of view?
% Where is the added value?
\section*{Executive Summary}	
	This project is a contribution to a law academic’s research into the growing relationship between intellectual 	property law and human rights law; in particular, the extent to which intellectual property laws involve human rights considerations, and their balance between consideration of creators and users. The first technical objective for the project is to use natural language processing to estimate measures of these two aspects for input intellectual property statutes. This will involve using Python to explore a variety of methods, such as k-means clustering and maximum likelihood, to optimise results. The second is to create the interface for this to allow the input of new data and visualisation of the results. 

Currently, there is limited means for doing this type of analysis because it has to be done manually, making it time consuming to analyse significant amounts of data. The project will change this by allowing the user to automatically process legal documents as they become available.  

The project is likely to be successful because there have been projects previously successfully completed that involved natural language processing on legal documents for other analysis. It will, however, be made individual from many of these by my iterative approach to the project, collaborating with a law academic to find the most appropriate way to classify and visualise the results. 

Should it be successful, the project will further understanding of the legal impact of intellectual property laws. 

\tableofcontents