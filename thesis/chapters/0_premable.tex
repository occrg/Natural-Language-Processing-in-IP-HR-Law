\section*{Acknowledgements}
	\newpage
	
% Background and motivation
% What is the problem? What do they do at the moment?
% Why is this a potential solution?
% What are the technical challenges from a Computer Science point of view?
% Where is the added value?
\section*{Executive Summary}	
	This project is a contribution to a law academic's research into the growing relationship between intellectual 	property law and human rights law; in particular, the extent to which intellectual property laws involve human rights considerations, and their balance between consideration of creators and users. 
	
	The first technical objective for the project is to use natural language processing to estimate measures of these two aspects for input law journals and statutes. This will involve using Python to explore a variety of machine learning models, starting with support vector machines, to optimise results. The second is to collaborate with the law acadmic iteratively to find the most appropriate way to visualise the results for the project's domain.

Currently, the relationship between intellectual property and human rights is analysed on a case-by-case basis due to a lack of systematic means of analysis. The project will allow large amounts of evidence.  

The project is likely to be successful because there have been previously successfully projects that involved natural language processing on legal documents for other analysis. It will, however, be made individual from many of these by my iterative approach to the project, collaborating with a law academic to find the most appropriate way to classify and visualise the results. 

Should it be successful, the project will further understanding of significant past events on intellectual property laws. 

\tableofcontents